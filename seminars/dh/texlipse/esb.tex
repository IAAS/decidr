%%%%%%%%%%%%%%%%%%%%% esb.tex %%%%%%%%%%%%%%%%%%%%%%%%%%%%%%%%%
%
% The main chapters.
%
%%%%%%%%%%%%%%%%%%%% Springer-Verlag template %%%%%%%%%%%%%%%%% 

\chapter{Enterprise Service Bus}
\label{chap:enterprise-service-bus}
In most organizations, technological heterogenity is more the rule than the
exception. While new IT applications are often built using promising approaches
such as service-oriented architecture (SOA), valuable business data remains
locked up within existing applications. The Enterprise Service Bus
(ESB) is an integration middleware that enables an SOA by coordinating the communication
between services and those existing applications.
There is no standardized definition of what comprises an ESB and what
doesn't\cite{wiki-de}. However, David Chappell, a major contributor to the
emergence of the ESB, gives us a concise set of capabilities that make the
basis of any good ESB: 

\begin{mycite}{chappell}
An ESB is a standards-based integration platform that
combines messaging, web services, data transformation and intelligent routing to
reliably connect and coordinate the interaction of significant numbers of
diverse applications \ldots with transactional integrity.
\end{mycite}


%\newpage
\section{ESB Tasks}
\label{sec:esb-tasks}
%% What makes an ESB useful?
%% - service orchestration
%% - location transparency
%% - even more buzzwords, make sure this doesn't turn into an advertisement!
What makes an ESB useful? Explain service orchestration, location transparency
and other buzzwords.

%\newpage
\subsection{ESB Usage in Decidr}
\label{subsec:esb-usage-in-decidr}
%% What are the requirements of the Decidr appliaction?
Which requirements justify the usage of an ESB in the Decidr application? 
Will Decidr benefit from an ESB?

\begin{quotation}\emph{
The ESB can be introduced at a departmental level or on a
per-project basis. Adopting the ESB at the project level
allows you to get accustomed to doing standards-based
integration using ESB service containers, with full confidence
that the project will fit into a larger integration network
\ldots}\cite{chappell}.
\end{quotation}

%\newpage
\subsubsection{Load Balancing and Service Availability}
\label{subsubsec:load-balancing-and-service-availability}
%%
What role will the ESB play in increasing robustness, response times and
availability of the Decidr application?

%\newpage
\subsubsection{Message Transformation / Protocol Transformation}
\label{subsubsec:message-transformation-protocol-transformation}
%%
MT / PT explained (again) in the Decidr context. If it turns out that there are
more tasks each will get its own subsection.

%\newpage
\section{A Lightweight ESB - Apache Synapse}
\label{sec:a-lightweight-esb-apache-synapse}

%% Assessment of the Synapse ESB. Will it fulfill the Decidr requirements?
An assessment of the Synapse ESB (preceded by a \emph{short} introduction).
(How) does it fulfill the requirements set by the Decidr application? 

%\newpage
\subsection{Deployment}
\label{subsec:deployment}
%% What do I have to do to install Synapse?
Describes the installation and configuration of Synapse.

%\newpage
\chapter{Example ESB Usage}
\label{chap:example-esb-usage}

%% Demonstration of Synapse using Reinhold's Web Services and Modood's
%% Client.
Demonstration of the Synapse ESB using Reinhold's web services and Modood's
client.

\section{Example 1: Foo}

\section{Example 2: Bar}

%\newpage
\chapter{Web Services Security}
\label{chap:web-services-security}
%% What the hell is WS-Security?
Explain what WS-Security is. Who developed it? What does it
specify? Explain how it is related to WS-* and web service
technology in general.

%\newpage
\section{Comparison to Transport Layer Security}
\label{sec:comparison-to-transport-layer-securtiy}
%% Can't we just use TLS instead?
What's the difference between using WS-Security and using HTTPS?

%\newpage
\subsection{WS-Security Usage in Decidr}
\label{subsec:ws-security-usage-in-decidr}
%% (Where, why) does Decdir require the usage of WSS?
Why and where does Decidr require the usage of WSS?

%\newpage
\subsection{Example WS-Security Usage}
\label{subsec:example-ws-security-usage}
%% Well it's in the message header, so I'm not really sure if this deserves its
%% own subsection.
Shows how WS-Security is applied to SOAP messages to provide end-to-end
security, referencing the chapter \ref{chap:example-esb-usage}. 
