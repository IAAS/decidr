%%%%%%%%%%%%%%%%%%%%% esb.tex %%%%%%%%%%%%%%%%%%%%%%%%%%%%%%%%%
%
% The main chapters.
%
%%%%%%%%%%%%%%%%%%%% Springer-Verlag template %%%%%%%%%%%%%%%%% 
\begin{abstract}
In most organizations, technological heterogenity is more the rule than the
exception. While new software applications are often built with integration in
mind and using promising approaches such as service-oriented architecture (SOA), valuable
business data remains locked up within existing applications. The Enterprise
Service Bus (ESB) is an integration middleware that enables an SOA by
coordinating the communication between services and those existing
applications. In this paper we'll show that despite being developed in a
homogenous SOA environment, the DecidR application can benefit from Apache
Synapse, a lightweight ESB. \todo{Say something about WS-Security}
\end{abstract}

\section{Enterprise Service Bus}
\label{chap:enterprise-service-bus}
As an organization or business grows, so does the number of software applications
that need to communicate with each other in order to exchange business data.
Unless integration is considered an important matter early, it is very likely
that an ``accidental architecture'' will emerge:

\img{figures/network-point-to-point.pdf}{The problem: point to point connectivity
between applications requires definition, implementation and maintenance of
$\mathcal{O}(n^2)$ interfaces.}{fig:network-point-to-point}

Enterprise Application Integration (EAI) is an attempt to reduce integration
complexity and costs by introducing a mediator. Applications connect to a central
integration broker via an adapter, which significantly reduces the
number of interfaces that need to be developed and maintained.

\img{figures/network-hub-and-spoke.pdf}{The hub-and-spoke integration network
reduces complexity to $\Theta(n)$ interfaces.}{fig:network-hub-and-spoke}

A major disadvantage of previous EAI implementations is that most rely on
proprietary, competing standards, making business-to-business communication
difficult. Also, all information must traverse the central hub, which prohibits
a highly distributed infrastructure on the side of the integration broker.
In contrast, a distinguishing key feature of an ESB is its
distributable infrastructure:

\img{figures/network-bus.pdf}{The bus topology is well suited for
highly distributed deployment.}{fig:network-bus}

\newpage
Unfortunately, there is no standardized definition of what comprises an ESB and
what doesn't\supercite{wiki-de}. Some comfort can be taken from the description of a
set of capabilities that make the basis of an ESB, given by David Chappell, a
major contributor to the emergence of the term: 

\begin{mycite}{chappell}
An ESB is a standards-based integration platform that
combines messaging, web services, data transformation and intelligent routing to
reliably connect and coordinate the interaction of significant numbers of
diverse applications \ldots with transactional integrity.
\end{mycite}

%\newpage
\subsection{ESB Tasks}
\label{sec:esb-tasks}

\subsubsection{Vetro pattern?}



\subsubsection{Mediation and Invocation}
\label{subsec:mediation-and-invocation}

\subsubsection{Content-based Routing}
\label{subsec:content-based-routing}

\subsubsection{Data transformation}
\label{subsec:data-transformation}

\subsubsection{Messaging}
\label{subsec:messaging}

\subsubsection{Quality of Service}
\label{subsec:quality-of-service}

\subsubsection{Service Orchestration}
\label{subsec:service-orchestration}



%\newpage
\subsection{ESB Usage in Decidr}
\label{subsec:esb-usage-in-decidr}
%% What are the requirements of the Decidr appliaction?
Which requirements justify the usage of an ESB in the Decidr application? 
Will Decidr benefit from an ESB?

\begin{mycite}{chappell}The ESB can be introduced at a departmental level or on
a per-project basis. Adopting the ESB at the project level
allows you to get accustomed to doing standards-based
integration using ESB service containers, with full confidence
that the project will fit into a larger integration network
\ldots
\end{mycite}

%\newpage
\subsubsection{Load Balancing and Service Availability}
\label{subsec:load-balancing-and-service-availability}
%%
What role will the ESB play in increasing robustness, response times and
availability of the Decidr application?

%\newpage
\subsubsection{Message Transformation / Protocol Transformation}
\label{subsec:message-transformation-protocol-transformation}
%%
MT / PT explained (again) in the Decidr context. If it turns out that there are
more tasks each will get its own subsection.

%\newpage
\subsection{A Lightweight ESB - Apache Synapse}
\label{sec:a-lightweight-esb-apache-synapse}
\todo{
An assessment of the Synapse ESB (preceded by a \emph{short} introduction).
(How) does it fulfill the requirements set by the Decidr application? 
}
%\newpage
\subsubsection{Deployment}
\label{subsec:deployment}
\todo{
Describes the installation and configuration of Synapse.}

%\newpage
\section{Example ESB Usage}
\label{chap:example-esb-usage}

\todo{Demonstration of the Synapse ESB using Reinhold's web services and
Modood's client.}

\subsection{Example 1: Foo}

\subsection{Example 2: Bar}

%\newpage
\section{Web Services Security}
\label{chap:web-services-security}
\todo{Explain what WS-Security is. Who developed it? What does it
specify? Explain how it is related to WS-* and web service
technology in general.}

%\newpage
\subsection{Comparison to Transport Layer Security}
\label{sec:comparison-to-transport-layer-securtiy}
\todo{What's the difference between using WS-Security and using HTTPS?}

%\newpage
\subsubsection{WS-Security Usage in Decidr}
\label{subsec:ws-security-usage-in-decidr}
\todo{Why and where does Decidr require the usage of WSS?}

%\newpage
\paragraph{Example WS-Security Usage}
\label{subsec:example-ws-security-usage}
\todo{Shows how WS-Security is applied to SOAP messages to provide end-to-end
security, referencing the chapter \ref{chap:example-esb-usage}. }
