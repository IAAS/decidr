\subsection{Funktionsweise des SVN und Tools}

Im \marginpar{Funktionsweise des SVN} vorigen Unterkapitel wurde erl�utert, wie das SVN funktioniert. In diesem Unterkapitel wird erkl�rt was f�r Funktionalit�ten SVN bietet, bzw. wie man mit SVN arbeitet. Anschlie�end werden Tools f�r die Arbeit mit SVN aufgelistet. Bei den Befehlen bezieht sich die angegebene Adresse auf das SVN-Repository vom Studienprojekt \glqq DecidR\grqq.

Nachdem \marginpar{Befehle} das SVN-Repository auf einem Server angelegt und die Ordnerstruktur gew�hlt wurde, muss der Entwickler die gleiche Ordnerstruktur auf seinem Betriebssystem (OS) herstellen. Dazu bietet SVN einen Befehl an:
\begin{verbatim}
 svn checkout http://decidr.googlecode.com/svn/trunk/ decidr
\end{verbatim}
Dieser Befehl berechtigt den Entwickler nur das Lesen der Ordnerstruktur. Falls man selber Dateien hochladen oder �nderungen vornehmen m�chte, muss das Checkout verschl�sselt geschehen:
\begin{verbatim}
 svn checkout https://decidr.googlecode.com/svn/trunk/ 
 decidr --username username
\end{verbatim}
Man muss dabei seinen Benutzernamen und sein Passwort eingeben, erst dann erfolgt der Checkout in den Ordner, den man nach der Repository-Adresse angegeben hat. Beim ersten Checkout wird in dem Ordner eine \texttt{.svn} Datei erstellt, in der die kompletten Daten des SVN-Repositorys stehen.

S�mtliche Updates auf dem Repository kann man mit dem Update-Befehl auf sein OS holen:
\begin{verbatim}
 svn update decidr
\end{verbatim}
Dabei reicht es, wenn man den Ordner angibt der erneuert werden soll. Dieser muss aber die \texttt{.svn} Datei enthalten.

Falls man neue Dateien in das Repository laden m�chte, muss man zun�chst die Dateien vormerken. Dies geschieht mit:
\begin{verbatim}
 svn add file
\end{verbatim}
Daraufhin muss dem Repository noch mitgeteilt werden, dass die markierten Dateien nun hochgeladen und ins Repository aufgenommen werden sollen:
\begin{verbatim}
 svn commit decidr
\end{verbatim}
Beim commit ist es �blich, dass man noch Kommentare mitschreibt, zum besseren Verst�ndnis der �nderungen. Das realisiert man mit dem Parameter -m \glqq kommentar\grqq.

Um Konflikte zu vermeiden, falls zwei Entwickler gleichzeitig an einer Datei arbeiten, kann man mit Hilfe von SVN Dateien sperren lassen:
\begin{verbatim}
 svn lock filename
\end{verbatim}
Diese m�ssen nach Beenden der �nderung wieder entsperrt werden:
\begin{verbatim}
 svn unlock filename
\end{verbatim}


\subsection{Tools f�r SVN}

F�r \marginpar{Tools} SVN gibt es eine Reihe von Tools. Hier werden welche f�r MacOS, Windows, Linux und plattformunabh�ngige Systeme vorgestellt.
Plattformunabh�ngige Tools sind Tools, die auf jeder Plattform benutzt werden k�nnen. Im Studienprojekt benutzen wir das plattformunabh�ngige Tool smartSVN. Wir haben dieses Tool ausgew�hlt, damit wir ein einheitliches Tool besitzen, da unsere Teammitglieder mit Windows, Linux und MacOS arbeiten. 

Hier einige plattformunabh�ngige Tools:
\begin{itemize}
 \item rapidSVN \cite{rapidSVN}
 \item subcommander \cite{subcommander}
 \item smartSVN \cite{smartsvn}
\end{itemize}

Ein Tool f�r Windows, welches nicht auf einer Applikation beruht, sondern in Form des Contextmen�s alle Funktionen verf�gbar macht:
\begin{itemize}
 \item TortoiseSVN \cite{tortoiseSVN} 
\end{itemize}

Ein Tool f�r MacOS:
\begin{itemize}
 \item versions \cite{versions} 
\end{itemize}

Ein Tool f�r Eclipse, welches wir benutzen, um den Source-Code direkt in Eclipse auschecken, bzw. updaten und committen zu k�nnen:
\begin{itemize}
 \item subclipse \cite{subclipse} 
\end{itemize}




