\section{Zusammenfassung}

In der Ausarbeitung habe ich drei g�ngige Werkzeuge beschrieben, die heutzutage als Standard gelten, wenn man gr��ere Projekte entwickeln m�chte.

Subversion ist nicht mehr als Werkzeug wegzudenken. Es bietet eine stabile Grundlage zur Sicherung und zur Verwaltung der Dateien. Dadurch kann eine gro�e Softwaregruppe gleichzeitig an einem Projekt arbeiten. Und durch das ausgepr�gte Loggen kann man stets nachvollziehen, welche �nderungen vorgenommen wurden. Zum anderen dient Subversion ebenfalls als Nachweis f�r die Arbeit eines Entwicklers. Durch weitere Programme kann man sich die kompletten svn-Daten auflisten lassen. Dabei sieht man welcher Entwickler welche Dateien hochgeladen hat. Ob der Inhalt der hochgeladenen Dateien sinnvoll ist, das kann Subversion nicht kontrollieren. Daher ist dieser Nachweis keine hundertprozentige Sicherheit daf�r, ob der Entwickler effektiv gearbeitet hat. Man kann es aber zur Bewertung hinzuziehen.

Maven erleichtert einige Aufgaben eines Entwicklers oder einer Entwicklergruppe. Der standardisierte Software-Lebenszyklus wird automatisiert und unterschiedliche Strukturen werden aufgehoben. Dadurch entsteht ein einheitliches Programm, das sp�ter ohne gro�e Probleme in anderen Projekten integriert werden kann. Dabei werden fr�hzeitig Fehler entdeckt, die Entwickler vor gro�en Problemen stellen k�nnten. In Kombination mit Hudson, das permanent bei neuen �nderungen die Anwendung neu baut und testet, ist Maven ein weiteres Tool, mit dem Entwickler sehr viel Zeit sparen k�nnen. Diese Zeit k�nnen die Programmierer in Phasen investieren, die der Planung des Projektes dienen (Projektplan, Spezifikation, Entwurf). Das gew�hrleistet eine bessere Planung mit hoher Qualit�t, wenigen Fehlern und einer stabilen Software. 

In Bezug auf das Studienprojekt \glqq DecidR\grqq{} wird Subversion bereits benutzt. Ob sich Maven und Hudson integrieren lassen wird sich entscheiden. Mein pers�nlicher Wunsch w�re es, diese Tools zu benutzen, da sie uns sehr viel Arbeit abbnehmen w�rden. Au�erdem w�rden sie uns zur fr�hen Fehlererkennung zwingen. Das erspart uns sehr viel Aufwand im Nachhinein bei der Fehlerbehebung und wir minimieren dadurch das Risiko des Scheiterns des Projekts