\section{Zusammenfassung}
Im der Ausarbeitung habe ich zwei g�ngige Tools beschrieben, die heutzutage als Standard gelten, wenn man gr��ere Projekte entwickeln m�chte.\\
Subversion ist nicht mehr vom Projektmanagement weg zu denken. Es bietet eine stabile Grundlage zur Sicherung der Dateien und zur Verwaltung der Dateien, sodass eine gro�e Softwaregruppe mit bis zu zehn Entwicklern gleichzeitig an einem Projekt arbeiten kann, ohne dass es zu Konflikten kommt. Und durch das ausgepr�gte Loggen kann man immer nachvollziehen, welche �nderunge vorgenommen wurden. Zum anderen dient Subversion ebenfalls als Nachweis. Durch weitere Programme kann man sich die kompletten svn-Daten auflisten lassen und kann �berpr�fen, welcher Entwickler wieviel hochgeladen hat. Die m�ssen nicht ausschlaggebend f�r eine Bewertung des Entwicklers sein, aber man kann sie in die Bewertung mit einbeziehen. Man sieht also, dass Subversion mehr zu bieten hat als nur eine Versionsverwaltung. \\
Maven erleichtert einige Aufgaben eines Entwicklers oder einer Entwicklergruppe. Der standardisierte Software-Lebenszyklus wird automatisiert und unterschiedliche Strukturen werden aufgehoben, sodass eine einheitliche Anwendung entsteht, die sp�ter ohne gro�e Probleme in anderen Projekten integriert werden kann. Dabei werden fr�hzeitig Fehler entdeckt, die Entwickler vor gro�en Problemen stellen k�nnten. In Kombination mit Hudson, der permanent bei neuen �nderungen die Anwendung neu baut und testet, ist Maven ein weiteres Tool f�r das Projektmanagement, mit dem Entwickler sehr viel Zeit sparen k�nnen und diese Zeit in Phasen investieren k�nnen, die der Planung des Projektes dienen (Projektplan, Spezifikation, Entwurf). Das gew�hrleistet eine bessere Planung mit hoher Qualit�t, wenigen Fehlern und einer stabilen Software. \\
In Bezug auf das Studienprojekt \flqq DecidR \frqq wird Subversion schon benutzt. Ob sich Maven und Hudson mitintegrieren lassen wird sich entscheiden. Meine pers�nlicher Wunsch w�re es, diese Tools zu benutzen.