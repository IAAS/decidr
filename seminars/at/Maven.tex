\section{Maven}
Maven kommt aus dem J�dischen (\emph{meyvn}) bzw. aus dem Hebr�ischem (\emph{mevin}) und bedeutet soviel wie Experte. Die genaue englische �bersetzung lautet accumulator of knowledge, quasi ein Speicher f�r Wissen. \\
Maven wurde als Ansatz im \emph{Apache Jakarta Turbine Project} \footnote{Turbine ist ein Servlet basiertes Framework f�r erfahrene Java-Entwickler, um schnell Web-Applikationen zu erstellen}  benutzt, um den Build-Prozess zu vereinfachen. Es gab einige verschiedene Projekte mit ihren eigenen Ant-Buildfiles, die alle unterschiedlich waren und die Jar-Dateien wurden alle ins CVS geladen.\\
Daraus resultierte der Wunsch eines einheitlichen automatisierten Verfahrens, um den kompletten Build-Prozess zu standardisieren. Dabei setzt sich Maven mit folgenden Zielen auseinander:
\begin{itemize}
 \item Vereinfachung des Build-Prozesses
 \item Bereitstellung eines einheitlichen Build-Systems
 \item Bereitstellung von Qualit�tsinformationen bez�glich des Projektes
 \item Bereitstellung von Richtlininen zur besten Entwicklung
 \item Transparente Migration von neuen Features (Pklug-In)\\
\end{itemize}
Au�erdem versucht Maven das Prinzip \emph{Convention over Configuration}\footnote{Das Ziel ist, die Zahl der Entscheidungen, die ein Entwickler trifft, zu verringern und dabei Einfachheit zu erreichen, ohne die Felxibilt�t zu beeinflussen} f�r den gesamten Software-Lebenszyklus zu verfolgen. \\
Die erste Version von Maven war noch stark an Ant angelehnt, sodass sich diese beiden Build-Methoden �hnelten. Die zweite Version verfolgte grundlegende neue Ans�tze und ist nicht mehr mit der vorherigen Version kompatibel.\\

\subsection{Die Philosophie Mavens}
\marginpar{pom.xml}Beim Erstellen eins Maven Projektes wird ein einheitliches Projektverzeichnis und eine einheitliche Projektstruktur verwendet. Maven erstellt eine XML-Datei, die sogenannte \emph{pom.xml}. In dieser stehen die Grunddaten f�r das Projekt. Zum Beispiel stehen in der Datei, der Name des Projektes, die Version, die Art des Paketes, das sp�ter erstellt werden soll, die URL, die sich auf das Projekt bezieht und die Abh�ngigkeiten, die das Projekt mit sich bringt und aufgel�st werden sollen, wenn das Projekt erstellt wird. \\
\marginpar{settings.xml}Zus�tzlich wird eine weitere Konfigurationsverwaltung erstellt, die \emph{settings.xml}. In ihr werden die Zugangsdaten f�r den Repository-Server und Deploy-Server gespeichert. Desweiteren werden die Einbindung von Plugins- oder Bibliotheken-Repositorys oder projekt�bergreifende Parameter eingestellt. Sowohl in der settings.xml als auch in der pom.xml gibt es sogennante Profile, die ein weiterer Mechanismus zur �nderung der Einstellungen sind. In ihr k�nnen abweichende Parameter gesetzt werden, wie z.B. f�r Filterdateien oder Repositorys.\\
\marginpar{.m2-Ordner}Wie schon erw�hnt werden bei der ersten Verwendung diese beiden XML-Dateien erstellt. Zus�tzlich wird noch ein weiterer Ordner erstellt. Dieser hei�t .m2 Ordner und in ihm befindet sich das lokale Repository, welches mit den Dateien des entfernten Maven-Repositorys gef�llt wird. Somit ist gew�hrleistet, dass bei einem erneuten Nutzen von Maven nicht alles aus dem entfernten Repository runtergeladen wird, sondern Maven greift nur noch auf das lokale zu, sofern die ben�tigten Pakete dort schon vorhanden sind. Wenn nicht, werden diese automatisch runtergeladen.\\
\marginpar{Zentrales Repository}Ein weiterer Grundgedanke besteht darin, s�mtliche Bibliotheken und Plugins an einem zentralen Ort zu verwalten. Dies vereinfacht bei vielen Projekten den Verwaltungsaufwand. Diese zentrale Repository bietet die Apache Group an. Man kann auch eigene Repositorys erstellen, der Vorteil besteht darin, dass man keine Internetverbindung ben�tigt und eine firmen�bergreifende Bibliothek erstellen kann. Der Vorteil des zentralen Repositorys liegt darin, dass Abh�ngigkeiten bereits abgelegt sind, und nicht manuell mit eingebaut werden m�ssen.\\
\marginpar{IDE} Es gibt f�r Eclipse und NetBeans Plugins, um Maven in den IDEs zu integrieren und dort auch zu nutzen. \\
Man sieht also, dass mit wenigen Konfigurationseinstellungen der Maven-Prozess ver�ndert werden kann. Aber diese Einstellungen �ndern nichts am automatisierten Vorgehen von Maven, dieser bleibt immer gleich. Das ist ein gro�er Vorteil. Den Maven baut seine Projekte durch das Project Object Model (POM) und somit ist ein einheitliches Build-System gew�hrleistet, was es wiederrum m�glich macht alle Maven Projekte zu verstehen. Dadurch spart man Zeit bei der Einarbeitung von anderen Projekten, da man die Bauweise kennt.\\
Maven basiert auf einer Plugin-Architektur, die es erlaubt verschiedene Plungins f�r verschiedene Anwendungen auf das Projekt anzuwenden, ohne diese explizit installieren zu m�ssen. Daher ist die Migration von Features sehr einfach gehalten und einfach m�glich. \\
\marginpar{Was Maven nicht ist!}Zum Schluss noch einige Dinge dar�ber was Maven nicht ist. Maven ist keine Erweiterung des Build-Tools Ant und ebenfalls nicht nur ausschlie�lich ein Dokumentations-Tool. Maven stellt die besten M�glichkeiten zur Verf�gung. Es kann durchaus sein, dass es manchmal nicht m�glich ist mit Maven ein Projekt zu bauen, weil die Struktur des Projektes so exotisch ist, dass eine Standardisierung gar nicht m�glich ist. Dann sollte man ,in diesem Fall, Maven lieber aufgeben und versuchen das Projekt auf eine andere Weise zu builden. \\
Beim kompletten Build-Prozess durchl�uft Maven verschiedene Phasen, in denen jeweils ein bestimmter Teil des Software-Lebenszyklus durchgearbeitet wird. Diese werden im n�chsten Abschnitt beschrieben.
