\subsection{Vergleich Maven, make und ant}

Maven ist das zweite Buildprojekt von Apache. Das erste Buildprojekt ist \emph{ant} \cite{ant}. Es existieren weitere Build-Tools, wovon \emph{make}  \cite{make} das prominenteste ist. \emph{Make} ist Teil des POSIX-Standards, dessen gegenw�rtige Bezeichnung IEEE Std 1003.1, 2004 Edition lautet.

Apache \marginpar{Apache-Ant} Ant ist ein Java-basiertes Werkzeug zur automatischen Erzeugung von Programmen aus Quelltext. Ant ist ein Akronym und steht f�r Another Neat Tool. Es wurde 1999 von James Duncan Davidson entwickelt, der ein Werkzeug wie make nur f�r Java ben�tigte. Ant wird durch eine einzige XML-Datei gesteuert, der sogenannten \texttt{build.xml}. In dieser Build-Datei wird ein Project ermittelt, welches das Wurzelelement darstellt. Desweiteren besitzt die \texttt{build.xml} sogenannte \emph{Targets}. Diese sind vergleichbar mit Funktionen in Programmiersprachen und k�nnen von au�en �ber die Kommandozeile aufgerufen werden. Abh�ngigkeiten, die in den Targets definiert werden, l�st Ant auf und dienen zum Bauen des Programmes. Da es sich um eine XML-Datei handelt ist diese nicht von Tabulatorzeichen oder sonstigen Zeichen abh�ngig. Dies ist eine Verbesserung gegen�ber dem alten Build-Tool make. 

Make \marginpar{make} liest ein sogenanntes \emph{Makefile}, in dem der �bersetzungsprozess von Programmen strukturiert erfasst ist, aus. Diese Formalisierung beschreibt, welche Quelltextdateien der Compiler zu welchen Objektdateien verarbeitet, und welche Objektdateien vom Linker dann zu Programmbibliotheken oder ausf�hrbaren Programmen verbunden werden. Alle Schritte erfolgen unter Beachtung der Abh�ngigkeiten, die m�glicherweise durch die Dateiorganisation gegeben sein k�nnen. Bei der Abarbeitung wird eine Umwandlung von einer Quelldatei nur dann vorgenommen, wenn die Quelldatei neuer als die Objektdatei ist. Der Vorteil dabei ist, dass bei kleinen Ver�nderungen in gro�en Projekten nicht die komplette Kompilation durchgef�hrt werden muss. Jedoch m�ssen alle Abh�ngigkeiten in der Makefile beschrieben werden, was bei sehr gro�en Projekten nicht immer leicht zu realisieren ist.

Der \marginpar{Vor- und Nachteile von Maven, ant und make} klare Vorteil bei Ant und Make ist die Vielzahl an Literatur, die es mittlerweile gibt und die praktische Erfahrung, da diese beiden Build-Tools am weitesten verbreitet sind. Au�erdem ist die Einarbeitung in Ant einfacher und es ist auch leichter zu handhaben. Die Einarbeitung in Maven f�llt wesentlicher gr��er aus. Daf�r bietet Maven eine einheitliche Projektstruktur und eine zentrale Verwaltung von Bibliotheken durch das Repository. Zudem ist es leicht erweiterbar, was f�r ant und make nicht gilt. Bei make spielen die Abh�ngigkeiten eine gro�e Rolle und k�nnen bei gro�en Projekten zu Problemen f�hren, um die sich Maven bei seinen Projekten nicht k�mmern muss. Einen weiteren Zusatz bietet das site von Maven, in der Informationen zum Projekt bereitgestellt werden. Maven und ant sind XML-basiert im Gegensatz zu make, welches einem Shellskript �hnelt.
