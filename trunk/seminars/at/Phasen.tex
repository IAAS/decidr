\subsection{Die Phasen Mavens}
\begin{center}
% use packages: array
\begin{tabular}[!t]{|p{6cm}|p{6cm}|}\hline
 Validierungsphase & Hier wird die pom.xml auf ihre Struktur �berpr�ft, ob diese korrekt ist. Erst wenn die korrekt ist, kann der Prozess weitergef�hrt werden \\ \hline
 Kompilierungsphase & Hier wird das Projekt kompiliert. Beim ersten Kompilieren, werden alle n�tigen Plugins heruntergeladen. Danach greift Maven auf das lokale System zu, falls es Plugins ben�tigt und sucht dort nach ihnen. Maven kompiliert die Klassen, die im Standard-Verzeichnis liegen. Die werden dort beim Erstellen des Projektes hineingelegt. \\ \hline
 Testphase & Hier werden die generierten Testklassen kompiliert und ausgef�hrt. Dabei werden zus�tzliche Plugins f�r das Kompilieren der Testklassen heruntergeladen. Bevor die Testklassen ausgef�hrt werden, wird der Maincode kompiliert. \\ \hline
 Paketierungsphase & In dieser Phase wird eine jar-Datei erstellt. In der pom.xml kann man angeben, was in dieser Phase erstellt werden soll. Die jar-Datei wird in einem standardisiertem Verzeichnis abgelegt. \\ \hline
 Integrations-Test & Hier wird durch ein Plugin das fertig Paket in eine Umgebung integriert und getestet. \\ \hline
 Verifizierungsphase & Hier wird das Paket �berpr�ft, ob es bestimmte Dateien/Ordner erh�lt und pr�ft den Inhalt auf ihre Struktur. \\ \hline
 Installationsphase & In dieser Phase wird das Paket, das im lokalen Maven-Repository liegt, installiert. \\ \hline
 Deployment & Ein Plugin f�r Maven sorgt daf�r, dass das erstellte Paket auf einem entfernten Repository gelegt wird, sodass andere Entwickler das Paket nutzen k�nnen. Alle n�tigen Informationen werden aus der pom.xml entnommen, sodass ein problemloses Deployment stattfinden kann. \\ \hline
 Site & Nun wird eine Seite erstellt, mit allen Informationen bez�glich des Projektes, sodass alle Informationen an einem Punkt sind. \\ \hline
 Clean & Clean l�scht den Zielordner mit all seinen Builddaten. \\ \hline
\end{tabular}
\end{center}
