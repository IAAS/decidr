\subsection{Organisation des SVN-Repositories}
Die Sturkturierung der Ordner ist wohl das wesentliche Hauptmerkmal des SVN. SVN kennt nur die Idee der Kopie und dementsprechend wird auch zur solch einer Ordnerstruktur geraten. Das bedeutet, man besitzt eine Hauptentwicklungslinie die den Stamm des kompletten Projektes bildet. Im Falle des Studienprojektes \flqq DecidR \frqq ist trunk der Hauptornder, indem das Hauptprojekt abgespeichert wird. Das es dabei sinnlos ist, alle Dateien nur in dem trunk Ordner zu legen ist eine weitere Untergliederung sehr sinnvoll. Es wurden die Ordner docs, prototype, seminars und src angelegt. Im Unterordner docs befinden sich alle Dokumente die in diesem Projekt erstellt werden (Angebot, Anforderungsanalyse, Spezifikation, Entwurf, Test, Handbuch usw.). F�r jedes weitere Dokument wurd ein weiterer Unterordner angelegt, indem dann die kompletten Dateien abgelegt werden. Der Unterordner prototype dient zur Ablage der Dateien f�r den Prototypen. Auch dieser enth�lt Unterordner, die sich speziell auf das Subprojekt beziehen. Der Ordner seminars bietet den Teammitgliedern die M�glichkeit ihr eigenes Seminar per Versionskontrolle zu erstellen. Jedes Teammitglied besitzt einen eigenen Ordner, den er selber weiter gestalten kann, wie er m�chte. Und im Unterordner src wird der Hauptcode abgelegt, der w�hrend der Implementierungsphase ensteht, abgelegt. Zudem gibt es noch den Ordner tags. Dieser Ordner ist dazu da, um alle Versionen festzuhalten, die als fertig erachtet werden. Das bedeutet, falls der erste Prototyp lauff�hig ist, wird die Version dort abgelegt, damit die Entwickler zu einem sp�teren Zeitpunkt genau wissen auf welche Version sie sich berufen m�ssen, falls sie etwas am fertig Prototypen �ndern m�chten. Nun kann man diskutieren inwieweit das sinnvoll ist, da SVN ja die Versionverwaltung anbietet und man dadurch auch auf fr�here Versionen zugreifen kann. Wir haben es als sinnvoll erachtet und dementsprechend den Ordner angelegt. Letztlich bleibt noch der Ordner branch. In diesem Ordner werden nach der Philosophie des SVN Kopien des Stammordners trunk angelegt, um, wenn evtl. n�tig, andere Entwicklungspfade einzuschlagen. Der Ordner ist somit ein Zweig der Hauptentwicklungslinie. Man kann dann andere Entwicklungpfade gehen, ohne den Stamm dabei zuverlieren. \\
Diese Ordnerstruktur wird als Standard angesehen und von allen SVN Betreibern empfohlen. Letzten Endes bleibt es dem Administrator �berlassen, wie er das SVN strukturiert.\\
Subversion kann im Gegensatz zu CVS, auch mit Bin�rdateien umgehen. Bei Bin�rdateien werden lediglich die �nderungen �bertragen. Jedoch lassen sich die Bin�rdateien nicht einfach so zusammenf�hren. Daher muss die zu bearbeitende Bin�rdatei gesperrt werden, bevor sie bearbeitet werden kann. Wenn die Bearbeitung fertig ist, wird sie wieder freigegeben und die �nderung wird beim n�chsten Update �bertragen.