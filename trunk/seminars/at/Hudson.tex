\subsection{Hudson continuous integration}
Hudson contiuous integration ist ein erweiterbares, webbasiertes Tool zur kontinuierlichen Integration in agilen Softwareprojekten. Kontinuierliche Integration bedeutet eine fortlaufende, permanente Integration der den Prozess des vollst�ndigen Neubildens und Testens einer Anwendung beschreibt. Dabei spielt das agile Softwareprojekt eine bedeutende Rolle. Denn durch die Agilit�t der Software wird die permanente Integration m�glich. Das Extreme Programming ist ein Beispiel f�r solch ein agiles Softwareprojekt. Sobald �nderungen in der Anwendung vorgenommen werden, wird die komplette Anwendung neu gebaut und automatisch getestet. Falls dieser Test erfolgreich ist, wird die Anwendung in die n�chste Stufe gereicht, falls nicht, findet ein Rollback statt und die Entwickler werden aufgefordert die Anwendung zu verbessern. Und genau das �bernimmt Hudson, in automatisierter Form. \\
\marginpar{Funktionalit�ten}Hudson wurde in erster Linie von Kohsuke Kawaguchi, Mitarbeiter von Sun Microsystems, entwickelt. Bis auf die Icons steht das komplette Programm unter der MIT-Lizenz. Hudson ist in Java geschrieben und l�uft auf einem beliebigen Servlet-Container. Es werden s�mtliche g�ngigen Build-Tools verwendet, wie Apache-Ant oder Apache-Maven. Zus�tzlich bietet Hudson noch die M�glichkeit der Versionsverwaltung (CVS oder Subversion). Desweiteren bietet Hudson die M�glichkeit �nderungen per Email zu verschicken und so die betreffenden Entwickler zu informieren. \\
\marginpar{Vorteile}Dabei steht die Einfachheit an erster Stelle. Gerade in Bezug auf Maven bietet Hudson eine sehr einfache und praktikable M�glichkeit. Es ist m�glich ein Maven Projekt zu erstellen, sodass Hudson es �berwacht und �nderungen zur sofortigen Neubildung und Testen der Anwendung f�hrt. Durch die kontinuierliche Integration, die Hudson bietet, wird die Programmierung vorteilhafter. Dabei werden
\begin{itemize}
 \item Fehler laufend entdeckt und behoben und nicht erst vor dem Meilenstein
 \item fr�he Warnungen bei nicht passenden Bestandteilen ausgegeben
 \item Fehler bei Unit-Tests sofort entdeckt
 \item lauff�hige Versionen f�r Test-, Demo- und Vertriebszwecke angeboten
 \item die Entwickler gezwungen verwantwortungsvoller mit ihren T�tigkeiten umzugehen, um Fehler zu vermeiden
\end{itemize}
