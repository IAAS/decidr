\subsection{Funktionsweise des SVN und Tools}
Im vorigen Unterkapitel wurde erl�utert, wie das SVN funktioniert. In diesem Unterkapitel wird erkl�rt was f�r Funktionalit�ten SVN bietet, bzw. wie man mit SVN arbeitet. Anschlie�end werden Tools f�r die Arbeit mit SVN aufgelistet. Bei den Befehlen bezieht sich die angegeben Adresse auf das SVN-Repository vom Studienprojekt \flqq DecidR \frqq.\\
Nachdem das SVN Repository auf einem Server angelegt und die Ordnerstruktur gew�hlt wurde, muss der Entwickler die gleiche Ordnerstruktur auf sein OS herstellen. Dazu bietet SVN einen Befehl an:
\begin{verbatim}
 svn checkout http://decidr.googlecode.com/svn/trunk/ decidr
\end{verbatim}
Dieser Befehl berechtigt den Entwickler nur das Lesen der Ordnerstruktur. Falls man selber Dateien hochladen oder �nderungen vornehmen m�chte, muss das Checkout verschl�sselt passieren:
\begin{verbatim}
 svn checkout https://decidr.googlecode.com/svn/trunk/ 
 decidr --username username
\end{verbatim}
Man muss dabei seinen Benutzernamen und sein Passwort eingeben, erst dann erfolgt der Checkout in den Ordner, den man nach der Repository-Adresse angegeben hat. Beim ersten checkout wird in dem Ordner eine .svn Datei erstellt, in der die kompletten Daten des SVN-Repositorys stehen.\\
S�mtliche Updates auf dem Repository kann mit dem Update-Befehl auf sein OS holen:
\begin{verbatim}
 svn update decidr
\end{verbatim}
Dabei reicht es, wenn man den Ordner angibt der erneuert werden soll. Dieser muss aber die .svn Datei erhalten.\\
Falls man neue Dateien in das Repository laden m�chte, muss man zun�chst die Dateien vormerken. Dies geschieht mit:
\begin{verbatim}
 svn add file
\end{verbatim}
Daraufhin muss dem Repository noch mitgeteilt werden, dass die markierten Dateien nun hochgeladen und ins Repository aufgenommen werden sollen:
\begin{verbatim}
 svn commit decidr
\end{verbatim}
Beim commit ist es �blich, dass man noch Kommentare mitschreibt, zum besseren Verst�ndnis was sich ge�ndert hat. Das erreicht man mit dem Parameter -m \flqq Kommentar \frqq.\\ \newpage



\vspace{2cm}

\textbf{Tools f�r SVN}
\begin{itemize}
 \item rapidSVN \cite{rapidSVN}
 \item subclipse \cite{subclipse}
 \item subcommander \cite{subcommander}
 \item TortoiseSVN \cite{tortoiseSVN}
 \item versions \cite{versions}
 \item statsvn \cite{statsvn}
 \item smartSVN \cite{smartsvn}
\end{itemize}



