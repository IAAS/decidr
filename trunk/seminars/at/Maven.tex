\section{Maven}

Der \marginpar{Definition} Begriff \glqq Maven\grqq{} selbst stammt aus dem J�dischen (\emph{meyvn}) bzw. aus dem Hebr�ischem (\emph{mevin}) und bedeutet soviel wie \glqq Experte\grqq{}. Die genaue englische �bersetzung lautet \glqq accumulator of knowledge\grqq{} \cite{maven}, quasi ein Speicher f�r Wissen. 

Maven wurde als Ansatz im \emph{Apache Jakarta Turbine Project} benutzt, um den Build-Prozess zu vereinfachen. Turbine ist ein Servlet basiertes Framework f�r erfahrene Java-Entwickler, um schnell Web-Applikationen zu erstellen. Es gab einige verschiedene Projekte mit ihren eigenen Ant-Buildfiles, die alle unterschiedlich waren und die Jar-Dateien wurden alle ins CVS geladen.

Daraus resultierte der Wunsch eines einheitlichen automatisierten Verfahrens, um den kompletten Build-Prozess zu standardisieren. Dabei setzt sich Maven mit folgenden Zielen auseinander:
\begin{itemize}
 \item Vereinfachung des Build-Prozesses
 \item Bereitstellung eines einheitlichen Build-Systems
 \item Bereitstellung von Qualit�tsinformationen bez�glich des Projektes
 \item Bereitstellung von Richtlininen zur besten Entwicklung
 \item Transparente Migration von neuen Features (mittels Plugins)
\end{itemize}

Au�erdem versucht Maven das Prinzip \emph{Convention over Configuration} \cite{convention} f�r den gesamten Software-Lebenszyklus zu verfolgen. Dabei ist das Ziel, die Zahl der Entscheidungen, die ein Entwickler trifft, zu verringern und dabei Einfachheit zu erreichen, ohne die Flexibilt�t zu beeinflussen.

Die erste Version von Maven war noch stark an Ant angelehnt, sodass sich diese beiden Build-Methoden �hnelten. Die zweite Version verfolgte grundlegende neue Ans�tze und ist nicht mehr mit der vorherigen Version kompatibel.

\subsection{Grunds�tze und Eigenschaften Mavens}
Ein \marginpar{Zentrales Repository} Grundgedanke besteht darin, s�mtliche Bibliotheken und Plugins an einem zentralen Ort zu verwalten. Dies vereinfacht bei vielen Projekten den Verwaltungsaufwand. Dieses zentrale Repository bietet die Apache Group an. Man kann auch eigene Repositorys erstellen. Der Vorteil besteht darin, dass man keine Internetverbindung ben�tigt und eine firmen�bergreifende Bibliothek erstellen kann. Der Vorteil des zentralen Repositorys liegt darin, dass Abh�ngigkeiten bereits abgelegt sind, und nicht manuell mit eingebaut werden m�ssen.

Beim \marginpar{pom.xml} Erstellen eines Maven Projektes kann der Benutzer die Ordnernamen und die Ordnertiefe genau festlegen. Dabei verfolgt Maven eine einheitliche Struktur. Java Klassen werden in der niedrigsten Ordnertiefe abgelegt. Maven bezieht sich beim Builden immer auf die unterste Stufe der Ordnerstruktur, sodass eine Automatisierung erm�glicht wird. Maven erstellt eine XML-Datei, die sogenannte \texttt{pom.xml}. In dieser stehen die Grunddaten f�r das Projekt, welche aus dem Project Object Model (POM) bezogen werden. Somit ist ein einheitliches Build-System gew�hrleistet, was es wiederrum m�glich macht alle Maven Projekte zu verstehen. Dadurch spart man Zeit bei der Einarbeitung von anderen Projekten, da man die Bauweise kennt. Zum Beispiel stehen in der \texttt{pom.xml}, der Name des Projektes, die Version, die Art des Paketes, das sp�ter erstellt werden soll, die URL, die sich auf das Repository bezieht und die Abh�ngigkeiten, die das Projekt mit sich bringt und aufgel�st werden sollen, wenn das Projekt erstellt wird. 

Zus�tzlich \marginpar{settings.xml} wird eine Konfigurationsverwaltung erstellt, die \texttt{settings.xml}. In ihr werden die Zugangsdaten f�r den Repository-Server gespeichert. Desweiteren werden die Einbindung von Plugins- oder Bibliotheken-Repositorys oder projekt�bergreifende Parameter eingestellt. Sowohl in der \texttt{settings.xml} als auch in der \texttt{pom.xml} gibt es sogennante Profile, die ein weiterer Mechanismus zur �nderung der Einstellungen sind. In ihr k�nnen abweichende Parameter gesetzt werden, wie z.B. f�r Filterdateien oder Repositorys.

Wie \marginpar{.m2-Ordner} schon erw�hnt, werden bei der ersten Verwendung diese beiden XML-Dateien erstellt. Zus�tzlich wird noch ein weiterer Ordner erstellt. Dieser hei�t \texttt{.m2} und in ihm befindet sich das lokale Repository, welches mit den Dateien des entfernten Maven-Repositorys gef�llt wird. Somit ist gew�hrleistet, dass bei einem erneuten Benutzen von Maven nicht alles aus dem entfernten Repository heruntergeladen wird. Maven greift nur noch auf das lokale Repository zu, sofern die ben�tigten Pakete dort schon vorhanden sind. Wenn nicht, werden diese automatisch heruntergeladen.

Es \marginpar{IDE} gibt f�r Eclipse \cite{maven-eclipse} und NetBeans \cite{netbeans} Plugins, um Maven in den IDEs zu integrieren und dort auch zu nutzen. 

Man sieht also, dass mit wenigen Konfigurationseinstellungen die einzelnen Maven-Phasen (vgl. Abschnitt 3.2) ver�ndert werden k�nnen. Aber diese Einstellungen �ndern nichts am automatisierten Vorgehen von Maven. Dieser bleibt immer gleich.  

Maven basiert auf einer Plugin-Architektur, die es erlaubt verschiedene Plugins f�r verschiedene Anwendungen auf das Projekt anzuwenden. Dadurch ist Maven sehr einfach erweiterbar.

Zum \marginpar{Was Maven nicht ist} Schluss noch einige Dinge dar�ber was Maven nicht ist. Maven ist keine Erweiterung des Build-Tools Ant. Maven stellt nach Sichten der Entwickler die besten M�glichkeiten zur Verf�gung. Es kann durchaus sein, dass es manchmal nicht m�glich ist mit Maven ein Projekt zu bauen, weil die Struktur des Projektes so exotisch ist, dass eine Standardisierung gar nicht m�glich ist. Dann sollte man -in diesem Fall- Maven aufgeben und versuchen das Projekt, auf eine andere Weise zu builden. 

Beim kompletten Build-Prozess durchl�uft Maven verschiedene Phasen, in denen jeweils ein bestimmter Teil des Software-Lebenszyklus durchgearbeitet wird. Dabei startet der Lebenszyklus bei Maven erst in der Implementierungsphase. Phasen, die vor der Implementierung stattfinden, wie Spezifikation und Entwurf, werden von Maven nicht ber�cksichtigt. Diese werden im folgenden Abschnitt beschrieben.
