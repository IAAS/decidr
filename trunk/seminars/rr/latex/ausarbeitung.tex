\documentclass[runningheads]{llncs}

% make a proper TOC despite springer's llncs
\setcounter{tocdepth}{2}
\makeatletter
\renewcommand*\l@author[2]{}
\renewcommand*\l@title[2]{}
\makeatletter

\usepackage{ucs}
\usepackage[utf8x]{inputenc}
\usepackage{ngerman}
\usepackage[ngerman]{babel}
\usepackage[T1]{fontenc}
\usepackage[pdftex]{graphicx}
\usepackage[
  pdftex,
  colorlinks=true,
  urlcolor=blue,
  linkcolor=black,
  pdfpagemode=UseNone,
  pdfstartview=FitH,
  pdftitle={Seminar-Ausarbeitung: Web Services},
  pdfauthor={Reinhold Rumberger},
  pdfpagelayout=OneColumn]{hyperref}
\usepackage{array}
\usepackage{tabulary}

\newcommand{\germanquote}[1]{\glqq{}#1\grqq{}}
\newcommand{\decidr}{DecidR}
\newcommand{\anntabwidth}{\textwidth}

\author{Reinhold Rumberger}
\tocauthor{\ }
\institute{Institute of Architecture of Application Systems (IAAS)
  \email{rumberrd@studi.informatik.uni-stuttgart.de}}
\title{Seminar-Ausarbeitung: Web Services}
\date{27.02.2009}

\begin{document}
%FIXME: add handlers

  \frontmatter
  \pagestyle{headings}

  % TODO: make better title page
  \maketitle
  \tableofcontents
  \mainmatter

  \nocite{bk_ralph}

  \titlerunning{Web Services - Einleitung}
  \label{abstract}
  \begin{otherlanguage}{english}
    \begin{abstract}
      Large present-day businesses often use service oriented architectures (SOAs) to implement their business processes. One way to implement a SOA is through web services. This paper will provide an introduction into the basics of web services. It will then focus on the most important technologies used to implement web services in the Java programming environment.
    \end{abstract}
  \end{otherlanguage}


  \label{intro}
  \section{Einleitung}
    In der heutigen Software-Welt wird zunehmend auf Service-Orientierte Architekturen (SOAs) gesetzt. Das bedeutet, dass Anwendungen aus einzelnen Diensten aufgebaut werden, die sehr lose gekoppelt sind und über eine Art Netzwerk miteinander kommunizieren können. Durch diese Eigenschaften lassen sich auf SOA basierende Anwendungen schnell und effizient an neue Marktbedingungen anpassen.

    SOAs können durch Web Services implementiert werden. Für Java-Umgebungen spielen hier vor allem JAX-WS 2.0 (Java API for XML-Based Web Services, JSR-224), WS-Metadata 2.0 (JSR-181) und JAXB 2.0 (Java Architecture for XML Binding, JSR-222) eine Rolle. Diese Spezifikationen werde den Kern der Web Services in \decidr{} bilden.


  \titlerunning{Web Services - SOA}
  \label{soa}
  \section{SOA}
  \nocite{wk_soa}
    Eine \germanquote{SOA} (Service Oriented Architecture) ist eine Software-Architektur, bei der loose gekoppelte Dienste miteinander kommunizieren. Die SOA definiert die verfügbaren Kommunikationsmethoden und die Methoden zum Auffinden von Diensten. So können verschiedene Dienste miteinander verbunden werden, um einen bestimmten Geschäftsprozess zu implementieren. Mehrere Dienste werden kombiniert um eine Applikation zu realisieren.

    Die loose Kopplung der Dienste macht die durch sie implementierten Applikationen sehr flexibel. So kann relativ schnell auf veränderte Umweltbedingungen reagiert werden. Wenn beispielsweise Dienste zufällig aus einer Menge geeigneter Dienste ausgewählt werden, kann man einer gesteigerten Nachfrage durch Hinzufügen weiterer Dienste begegnen. Dabei kann es sich sowohl um neue Instanzen vorhandener Dienste handeln, als auch um komplett neu entwickelte Dienste.


  \titlerunning{Web Services - Web Services Grundlagen}
  \label{ws}
  \section{Web Services Grundlagen}
  \nocite{wk_ws}
    \germanquote{Web Service} wird vom W3C folgendermaßen definiert:
    \begin{quote}
      A Web service is a software system designed to support interoperable machine-to-machine interaction over a network. It has an interface described in a machine-processable format (specifically WSDL).\cite{w3c_wsgloss_ws}
    \end{quote}

    Es gibt zur Zeit zwei Ausprägungen von Web Services: die \germanquote{message-oriented Web Services} und die \germanquote{RESTful Web Services}.

    Die erste Ausprägung -- message-oriented -- benutzt Nachrichten zur Kommunikation zwischen den verschiedenen Diensten. So entsteht eine loose Kopplung, da nur eine von der Implementierung unabhängige Schnittstelle nach außen hin sichtbar ist. Sie werden auch \germanquote{Big Web Services} genannt. %TODO elaborate

    Die zweite Ausprägung -- RESTful -- versucht HTTP zu emulieren. Hier wird die Schnittstelle des Web Services auf wenige, häufig genutzte Operationen (z.B. GET, PUT, DELETE) beschränkt. Diese Web Services sind meist zustandsabhängig. Durch die Beschränkung der Operationen ist eine bessere Integration mit dem HTTP-Protokoll möglich. %TODO elaborate

    %TODO: find out about WS-I, was sind zusätzliche Einschränkungen (Beispiele)
    Um die Interoperabilität von Web Services zu steigern, publiziert das WS-I\cite{wsi_hp} Profile. Profile bestehen aus bestimmten Spezifikationen (z.B. SOAP, WSDL) in spezifischen Versionen (z.B. SOAP 1.2, WSDL 2.0). Hinzu kommen noch zusätzliche Einschränkungen. Außderdem publiziert das WS-I Anwendungsfälle und Testwerkzeuge um beim Deployen profilkonformer Web Services zu helfen.

    Es gibt zusätzlich Standards, die die Fähigkeiten von Web Services erweitern. Diese Standards haben gewöhnlich einen Namen nach dem Schema \germanquote{WS-x} (z.B. WS-Security, WS-Transaction). %TODO elaborate

  \label{soap}
  \subsection{SOAP}
  \nocite{wk_soap}
    %TODO: short intro, more detailed description
    %TODO: not transport protocol, actions
    SOAP ist ein Protokoll zum Austausch strukturierter Daten zwischen Web Services in Computernetzwerken. Es benutzt XML als Nachrichtenformat und stützt sich auf andere Protokolle der Anwendungsschicht (z.B. HTTP, SMTP, FTP) für dem Datentransport. SOAP stellt die Basis der Kommunikation zwischen Web Services dar.

    Die Wahl von XML als Nachrichtenformat hat sowohl Vor- als auch Nachteile.
    % FIXME: bad scentence! very bad scentence! no cookie for you!
    Positiv sind die leichtere Lesbarkeit für Menschen, die bessere Interoperabilität und die vereinfachte Fehlersuche. Allerdings ist es etwas unhandlich und verlangsamt die Verarbeitungsgeschwindigkeit durch seine Verbosität.

    Bei der Wahl des Transportprotokolls ist Vorsicht geboten; einige Protokolle werden typischerweise von Firewalls gefiltert und sollten deshalb gemieden werden.

  \subsection{WSDL}
  \label{wsdl}
  \nocite{wk_wsdl}
    Dieser Abschnitt setzt Wissen der Vorlesung \germanquote{Workflow Management}\cite{wfm_site} voraus. Konkret handelt es sich bei diesem Wissen um die im Foliensatz \germanquote{Workflow, Objects and Web Services}\cite{wfm_ch7} verwendeten Begriffe (Port, Binding, etc.). Die Folie 29 -- Terminology -- ist sehr hilfreich.

    WSDL (Web Services Description Language) ist eine XML-basierte Sprache zur Beschreibung der öffentlichen Schnittstelle von Web Services.
    %TODO: describe parts of WSDL

%     %TODO: label as example
%     \newpage
%     \label{ex_wsdl}
%     \begin{verbatim}
% <?xml version="1.0" encoding="UTF-8"?>
% <wsdl:definitions
%  xmlns:wsdl="http://schemas.xmlsoap.org/wsdl/"
%  xmlns:ns="http://decidr.org/mailws"
%  xmlns:wsaw="http://www.w3.org/2006/05/addressing/wsdl"
%  xmlns:xs="http://www.w3.org/2001/XMLSchema"
%  xmlns:soap="http://schemas.xmlsoap.org/wsdl/soap12/"
%  targetNamespace="http://decidr.org/mailws">
%   <wsdl:documentation>
%     An e-mail sending web service for the Decidr
%     prototype
%   </wsdl:documentation>
%
%   <!-- Abstract types -->
%   <wsdl:types>
%     <xs:schema attributeFormDefault="qualified"
%      elementFormDefault="qualified"
%      targetNamespace="http://decidr.org/mailws">
%       <xs:element name="sendEmail">
%         <xs:complexType>
%           <xs:sequence>
%             <xs:element minOccurs="0" maxOccurs="1" name="subject"
%              nillable="true" type="xs:string" />
%             <xs:element minOccurs="1" maxOccurs="1" name="message"
%              type="xs:string" />
%             <xs:element minOccurs="1" maxOccurs="1"
%              name="recipient" type="xs:string" />
%             <xs:element minOccurs="0" maxOccurs="1" name="sender"
%              nillable="true" type="xs:string" />
%           </xs:sequence>
%         </xs:complexType>
%       </xs:element>
%       <xs:element name="sendEmailResponse">
%         <xs:complexType>
%           <xs:sequence>
%             <xs:element minOccurs="1" maxOccurs="1" name="return"
%              type="xs:boolean" />
%           </xs:sequence>
%         </xs:complexType>
%       </xs:element>
%     </xs:schema>
%   </wsdl:types>
%
%   <wsdl:message name="sendEmailRequest">
%     <wsdl:part name="parameters" element="ns:sendEmail" />
%   </wsdl:message>
%   <wsdl:message name="sendEmailResponse">
%     <wsdl:part name="parameters" element="ns:sendEmailResponse" />
%   </wsdl:message>
%
%   <wsdl:portType name="eMailWSPortType">
%     <wsdl:operation name="sendEmail">
%       <wsdl:input message="ns:sendEmailRequest"
%        wsaw:Action="urn:sendEmail" />
%       <wsdl:output message="ns:sendEmailResponse"
%        wsaw:Action="urn:sendEmailResponse" />
%     </wsdl:operation>
%   </wsdl:portType>
%
%   <!-- Concrete Bindings -->
%   <wsdl:binding name="eMailWSSoap12Binding"
%    type="ns:eMailWSPortType">
%     <soap:binding
%      transport="http://schemas.xmlsoap.org/soap/http"
%      style="document" />
%     <wsdl:operation name="sendEmail">
%       <soap:operation soapAction="urn:sendEmail"
%        style="document" />
%       <wsdl:input>
%         <soap:body use="literal" />
%       </wsdl:input>
%       <wsdl:output>
%         <soap:body use="literal" />
%       </wsdl:output>
%     </wsdl:operation>
%   </wsdl:binding>
%
%   <!-- TODO: Adapt the following for deployment -->
%   <wsdl:service name="eMailWS">
%     <wsdl:port name="eMailWSHttpSoap12Endpoint"
%      binding="ns:eMailWSSoap12Binding">
%       <soap:address
%        location="http://localhost/eMailWS.Soap12Endpoint/" />
%     </wsdl:port>
%   </wsdl:service>
% </wsdl:definitions>
%     \end{verbatim}

  \label{uddi}
  \subsection{UDDI}
  \nocite{wk_uddi}


  \titlerunning{Web Services - Java und Web Services}
  \label{wsj}
  \section{Java und Web Services}

  \label{jsr224}
  \subsection{JAX-WS 2.0 (JSR-224)}
    Bei JAX-WS\cite{jsr_224} (Java API for XML-Based Web Services) handelt es sich um den Nachfolger von JAX-RPC (JSR-101). JAX-RPC war für RPC-orientierte Web Services konzipiert.

    JAX-WS definiert unter anderem Standard WSDL 1.1 $\Leftrightarrow$ Java Mappings, Standard SOAP- und HTTP-Bindings, ein Standard Handler-Framework und die Client-, Server und Core-APIs für JAX-WS-konforme Web Service-Im\-ple\-men\-tier\-ung\-en.

    \subsubsection{@WebServiceProvider}\ \\
    \tymin=75pt
    \begin{tabulary}{\anntabwidth}{|l|L|L|}
    \hline
    \textbf{Parameter} & \textbf{Zweck} & \textbf{Standard} \\
%     \hline
%       name &
%       Wird als \texttt{name}-Attribut im \texttt{wsdl:portType}-Element des generierten WSDL-\linebreak[0]Dokuments genutzt. &
%       Name der Klasse/\linebreak[0]des Interface \\
    \hline
      serviceName &
      Wird als \texttt{name}-Attribut im \texttt{wsdl:service}-Element des generierten WSDL-\linebreak[0]Dokuments genutzt. &
      \germanquote{} \\
    \hline
      portName &
      Wird als \texttt{name}-Attribut im \texttt{wsdl:port}-Element des generierten WSDL-\linebreak[0]Dokuments genutzt. &
      \germanquote{} \\
    \hline
      targetNamespace &
      Wird im generierten WSDL-Dokument als \texttt{targetNamespace} verwendet. &
      \germanquote{} \\
    \hline
      wsdlLocation &
      Eine URL, die auf eine vordefinierte WSDL-Datei zeigt. Die URL relativ oder absolut sein. Falls Inkonsistenzen zwischen der Implementierung und dem WSDL-Dokument bestehen, wird zur Laufzeit darauf hingewiesen. &
      \germanquote{} \\
    \hline
    \end{tabulary}
    \tymin=10pt

  \label{jsr181}
  \subsection{WS-Metadata 2.0 (JSR-181)}
  \nocite{jsr_181}
    % TODO: content

    Es folgt eine Liste der wichtigsten Annotationen mit ihren wichtigsten Parametern und deren Standard-Werten. Die Beschreibungen sind gekürzt aus der JSR-\linebreak[0]181-\linebreak[0]Spezifikation\cite{jsr_181} übernommen, weshalb teilweise Informationen fehlen. Genauere Details sind in der JSR-\linebreak[0]181-\linebreak[0]Spezifikation\cite{jsr_181} zu finden.\\ \vfill

    \subsubsection{@WebService}\ \\
      Markiert Klassen als Web Services oder Java Interfaces als Web Service Interfaces.\\
    \begin{tabulary}{\anntabwidth}{|l|L|L|}
    \hline
    \textbf{Parameter} & \textbf{Zweck} & \textbf{Standard} \\
    \hline
      name &
      Wird als \texttt{name}-Attribut im \texttt{wsdl:portType}-Element des generierten WSDL-\linebreak[0]Dokuments genutzt. &
      Name der Klasse/\linebreak[0]des Interface \\
    \hline
      serviceName &
      Wird als \texttt{name}-Attribut im \texttt{wsdl:service}-Element des generierten WSDL-\linebreak[0]Dokuments genutzt. Darf nicht in Endpoint Interfaces verwendet werden. &
      Name der Klasse + \germanquote{Ser\-vice} \\
    \hline
      portName &
      Wird als \texttt{name}-Attribut im \texttt{wsdl:port}-Element des generierten WSDL-\linebreak[0]Dokuments genutzt.\newline Darf nicht in Endpoint Interfaces verwendet werden. &
      \texttt{@WebService.name} + \germanquote{Port} \\
    \hline
      targetNamespace &
      Wird im generierten WSDL-Dokument als \texttt{targetNamespace} verwendet. Für Details bei welchen Elementen dieses Attribut gesetzt wird, siehe JSR-181\cite{jsr_181} &
      Im\-ple\-men\-tier\-ungs\-ab\-häng\-ig, siehe JAX-WS\cite{jsr_224}, Abschnitt 3.2 \\
    \hline
      wsdlLocation &
      Eine URL, die auf eine vordefinierte WSDL-Datei zeigt. Die URL relativ oder absolut sein. Falls Inkonsistenzen zwischen der Implementierung und dem WSDL-Dokument bestehen, wird zur Laufzeit darauf hingewiesen. &
      nichts \\
    \hline
      endpointInterface &
      Gibt den Namen des Service Endpoint Interfaces an, das implementiert werden soll. \newline Darf nicht in Endpoint Interfaces verwendet werden. &
      nichts\newline \newline Wird bei Bedarf im\-ple\-men\-tier\-ungs\-ab\-häng\-ig generiert, wenn es die Zielumgebung erfordert. \\
    \hline
    \end{tabulary} \vfill

    \subsubsection{@WebMethod}\ \\
      Bestimmt ob und mit welchen Eigenschaften eine Methode veröffentlicht wird.\\
    \tymin=75pt
    \begin{tabulary}{\anntabwidth}{|l|L|L|}
    \hline
    \textbf{Parameter} & \textbf{Zweck} & \textbf{Standard} \\
    \hline
      operationName &
      Wird als \texttt{name}-Attribut des \texttt{wsdl:operation}-Elements genutzt, das der annotierten Methode entspricht. &
      Name der annotierten Methode. \\
    \hline
      exclude &
      Dieser Parameter gibt an, dass die annotierte Methode \emph{nicht} veröffentlicht wird. \newline Wenn dieser Parameter auf \texttt{true} gesetzt wird, dürfen keine anderen Parameter gesetzt werden. &
      \texttt{false} \\
    \hline
    \end{tabulary}
    \tymin=10pt

    \subsubsection{@Oneway}\ \\
      Eine mit \texttt{@Oneway} annotierte \texttt{@WebMethod} hat keine Output-Message. Das bedeutet, dass der return-Typ \texttt{void} sein muss. Wenn eine mit \texttt{@Oneway} annotierte Methode einen Rückgabewert, definierte Exceptions oder OUT- bzw. INOUT-Parameter hat, muss spätestens zur Laufzeit ein Fehler gemeldet werden.


    \subsubsection{@WebParam}\ \\
      Bestimmt die Eigenschaften, mit denen ein Parameter veröffentlicht wird.\\
    \begin{tabulary}{\anntabwidth}{|l|L|L|}
    \hline
    \textbf{Parameter} & \textbf{Zweck} & \textbf{Standard} \\
    \hline
      name &
      Name des Parameters. Muss in bestimmten Umständen angegeben werden. &
      Normalerweise \texttt{arg\textit{N}}, wobei \texttt{\textit{N}} ein Integer $\geq$ 0 ist. \\
    \hline
      partName &
      Der name des \texttt{wsdl:part}s, das den Parameter repräsentiert. Wird nur in bestimmten Umständen benutzt. &
      \texttt{@WebParam.name} \\
    \hline
      targetNamespace &
      Der XML-Namespace des Parameters. Wird nur in bestimmten Umständen benutzt. &
      Entweder der leere Namespace oder der Standard-targetNamespace. \\
    \hline
      mode &
      Gibt an, ob der Parameter als Eingabe, Ausgabe oder beides dient. &
      INOUT, wenn der Parameter von \texttt{javax.xml.ws.Holder<T>} abgeleitet ist, sonst IN. \\
    \hline
      header &
      Wenn der Parameter im Nachrichtenkopf abgelegt ist, \texttt{true}, andernfalls \texttt{false}. &
      \texttt{false} \\
    \hline
    \end{tabulary}

    \subsubsection{@WebResult}\ \\
      Bestimmt die Eigenschaften, mit denen der Return-Wert einer Methode veröffentlicht wird. Diese Annotation wird auf Methoden angewendet.\\
    \begin{tabulary}{\anntabwidth}{|l|L|L|}
    \hline
    \textbf{Parameter} & \textbf{Zweck} & \textbf{Standard} \\
    \hline
      name &
      Name der Ausgabe. Muss in bestimmten Umständen angegeben werden. &
      Normalerweise \germanquote{return}. \\
    \hline
      partName &
      Der name des \texttt{wsdl:part}s, das die Ausgabe repräsentiert. Wird nur in bestimmten Umständen benutzt. &
      \texttt{@WebResult.name} \\
    \hline
      targetNamespace &
      Der XML-Namespace der Ausgabe. Wird nur in bestimmten Umständen benutzt. &
      Entweder der leere Namespace oder der Standard-targetNamespace. \\
    \hline
      header &
      Wenn die Ausgabe im Nachrichtenkopf abgelegt ist, \texttt{true}, andernfalls \texttt{false}. &
      \texttt{false} \\
    \hline
    \end{tabulary}

    \subsubsection{@HandlerChain}\ \\
    %TODO: research handler chain & fill in
    \begin{tabulary}{\anntabwidth}{|l|L|L|}
    \hline
    \textbf{Parameter} & \textbf{Zweck} & \textbf{Standard} \\
    \hline
      File &
      %TODO: research handler chain & fill in
      × &
      nichts \\
    \hline
    \end{tabulary}

    \subsubsection{@SOAPBinding}\ \\
      Spezifiziert, dass der Web Service auf das SOAP-Protokoll gemappt wird. Wird als \texttt{@SOAPBinding.Style} \germanquote{DOCUMENT} verwendet, dürfen Klassen und Methoden annotiert werden, andernfalls nur Klassen. Methoden, die nicht annotiert sind, verwenden die für die Klasse definierten Werte. \\
    \tymin=75pt
    \begin{tabulary}{\anntabwidth}{|l|L|L|}
    \hline
    \textbf{Parameter} & \textbf{Zweck} & \textbf{Standard} \\
    \hline
      Style &
      Der Kodierungsstil für Nachrichten von und zum Web Service. Entweder \germanquote{DOCUMENT} oder \germanquote{RPC}. &
      DOCUMENT \\
    \hline
      Use &
      Der Formattierungsstil für Nachrichten von und zum Web Service. Entweder \germanquote{LITERAL} odr \germanquote{ENCODED}. &
      LITERAL \\
    \hline
      parameterStyle &
      Gibt an, ob die Parameter den gesamten Inhalt der Nachricht wiedergeben (\germanquote{BARE}), oder ob sie in ein Element eingepackt werden (\germanquote{WRAPPED}). &
      WRAPPED \\
    \hline
    \end{tabulary}
    \tymin=10pt



  \label{jsr222}
  \subsection{JAXB 2.0 (JSR-222)}
  JAXB\cite{jsr_222} definiert Methoden zum Binding von Java-Klassen auf generierte XML-Schemas.

  \label{summary}
  \titlerunning{Zusammenfassung}
  \section{Zusammenfassung}
  In \decidr{} sollten wir folgende Spezifikationen in den angegebenen Versionen verwenden:
  \begin{itemize}
    \item JAX-WS 2.0
    \item JAXB 2.0
    \item SOAP 1.2
    \item WS-Metadata 2.0
    \item WSDL 1.1
  \end{itemize}


  \newpage
  \titlerunning{Zusammenfassung}
  \addcontentsline{toc}{section}{Literatur}
  \bibliography{literatur}
  \bibliographystyle{splncs}


\end{document}
