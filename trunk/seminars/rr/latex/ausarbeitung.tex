\documentclass[runningheads]{llncs}

% make a proper TOC despite springer's llncs
\setcounter{tocdepth}{2}
\makeatletter
\renewcommand*\l@author[2]{}
\renewcommand*\l@title[2]{}
\makeatletter

\usepackage{ucs}
\usepackage[utf8x]{inputenc}
\usepackage{ngerman}
\usepackage[ngerman]{babel}
\usepackage[T1]{fontenc}
\usepackage[pdftex]{graphicx}
\usepackage[pdftex, colorlinks=true, urlcolor=blue, linkcolor=black]{hyperref}
\usepackage{array}

\newcommand{\germanquote}[1]{\glqq{}#1\grqq{}}
\newcommand{\decidr}{Decidr}

\author{Reinhold Rumberger}
\tocauthor{\ }
\institute{Institute of Architecture of Application Systems (IAAS)
  \email{rumberrd@studi.informatik.uni-stuttgart.de}}
\title{Seminar-Ausarbeitung: Web Services}
\date{27.02.2009}

\begin{document}
%FIXME: add handlers

  \frontmatter
  \pagestyle{headings}

  % TODO: make better title page
  \maketitle
  \tableofcontents
  \mainmatter

  \titlerunning{Web Services - Einleitung}
  \label{intro}
  \section{Einleitung}
    Dies ist die Ausarbeitung des Seminars \germanquote{Web Services}. Ich werde hier kurz darauf eingehen, was eine Service-orientierte Architektur ist. Danach werde ich die Grundlagen der Web Services erläutern und besonders detailliert auf WSDL und SOAP eingehen. Zum Schluss erläutere ich die Implementierung und Verwendung von Web Services in Java. Dabei werde ich gesondert auf JAX-WS 2.0 (JSR-224), WS-Metadata 2.0 (JSR-181) und JAXB 2.0 (JSR-222) eingehen. In diesem Seminar konzentriere ich mich auf Aspekte von Web Services, die für die Entwicklung von \decidr{} relevant sind. Zudem werde ich mich besonders auf relevante Annotations konzentrieren.

    Zunächst sei aber noch erwähnt, dass \germanquote{JSRs} -- Java Specification Requests -- im Rahmen des Java Community Process erstellte Java-Spezifikationen sind.


  \titlerunning{Web Services - SOA}
  \label{soa}
  \section{SOA}
  \nocite{wk_soa}
    Eine \germanquote{SOA} (Service Oriented Architecture) ist eine Software-Architektur, bei der loose gekoppelte Dienste miteinander kommunizieren. Die SOA definiert die verfügbaren Kommunikationsmethoden und die Methoden zum Auffinden von Diensten. So können verschiedene Dienste miteinander verbunden werden, um einen bestimmten Geschäftsprozess zu implementieren. So werden Applikationen aus Diensten zusammengestellt.

    Die loose Kopplung der Dienste macht die auf ihnen implementierten Applikationen sehr flexibel. So kann relativ schnell auf veränderte Umweltbedingungen reagiert werden. Wenn beispielsweise Dienste zufällig aus einer Menge geeigneter Dienste ausgewählt werden, kann man auf eine gesteigerte Nachfrage durch Hinzufügen weiterer Dienste begegnen. Dabei kann es sich sowohl um neue Instanzen vorhandener Dienste handeln als auch um komplett neu entwickelte Dienste.


  \titlerunning{Web Services - Web Services Grundlagen}
  \label{ws}
  \section{Web Services Grundlagen}
  \nocite{wk_ws}
    \germanquote{Web Service} wird vom W3C folgendermaßen definiert:
    \begin{quote}
      A Web service is a software system designed to support interoperable machine-to-machine interaction over a network. It has an interface described in a machine-processable format (specifically WSDL). Other systems interact with the Web service in a manner prescribed by its description using SOAP-messages, typically conveyed using HTTP with an XML serialization in conjunction with other Web-related standards.\cite{w3c_wsgloss_ws}
    \end{quote}

    Es gibt zur Zeit zwei Ausprägungen von Web Services: die \germanquote{message-oriented Web Services} und die \germanquote{RESTful Web Services}.

    Die erste Ausprägung -- message-oriented -- benutzt Nachrichten zur Kommunikation zwischen den verschiedenen Diensten. So entsteht eine loose Kopplung, da nur eine von der Implementierung unabhängige Schnittstelle nach außen hin sichtbar ist. Sie werden auch \germanquote{Big Web Services} genannt. %TODO elaborate

    Die zweite Ausprägung -- RESTful -- versucht HTTP zu emulieren. Hier wird die Schnittstelle des Web Services auf wenige, häufig genutzte Operationen (z.B. GET, PUT, DELETE) beschränkt. Diese Web Services sind meist zustandsabhängig. Durch die Beschränkung der Operationen ist eine bessere Integration mit dem HTTP-Protokoll möglich. %TODO elaborate

    %TODO: find out about WS-I, was sind zusätzliche Einschränkungen (Beispiele)
    Um die Interoperabilität von Web Services zu steigern, publiziert das WS-I\cite{wsi_hp} Profile. Profile bestehen aus bestimmten Spezifikationen (z.B. SOAP, WSDL) in spezifischen Versionen (z.B. SOAP 1.2, WSDL 2.0). Hinzu kommen noch zusätzliche Einschränkungen. Außderdem publiziert das WS-I Anwendungsfälle und Testwerkzeuge um beim Deployen profilkonformer Web Services zu helfen.

    Es gibt zusätzlich Standards, die die Fähigkeiten von Web Services erweitern. Diese Standards haben gewöhnlich einen Namen nach dem Schema \germanquote{WS-x} (z.B. WS-Security, WS-Transaction). %TODO elaborate

  \subsection{WSDL}
  \label{wsdl}
  \nocite{wk_wsdl}
    Dieser Abschnitt setzt Wissen der Vorlesung \germanquote{Workflow Management}\cite{wfm_site} voraus. Konkret handelt es sich bei diesem Wissen um die im Foliensatz \germanquote{Workflow, Objects and Web Services}\cite{wfm_ch7} verwendeten Begriffe (Port, Binding, etc.). Die Folie 29 -- Terminology -- ist sehr hilfreich.

    WSDL (Web Services Description Language) ist eine XML-basierte Sprache zur Beschreibung der öffentlichen Schnittstelle von Web Services.
    %TODO: describe parts of WSDL

    %TODO: label as example
    \newpage
    \label{ex_wsdl}
    \begin{verbatim}
<?xml version="1.0" encoding="UTF-8"?>
<wsdl:definitions
 xmlns:wsdl="http://schemas.xmlsoap.org/wsdl/"
 xmlns:ns="http://decidr.org/mailws"
 xmlns:wsaw="http://www.w3.org/2006/05/addressing/wsdl"
 xmlns:xs="http://www.w3.org/2001/XMLSchema"
 xmlns:soap="http://schemas.xmlsoap.org/wsdl/soap12/"
 targetNamespace="http://decidr.org/mailws">
  <wsdl:documentation>
    An e-mail sending web service for the Decidr
    prototype
  </wsdl:documentation>

  <!-- Abstract types -->
  <wsdl:types>
    <xs:schema attributeFormDefault="qualified"
     elementFormDefault="qualified"
     targetNamespace="http://decidr.org/mailws">
      <xs:element name="sendEmail">
        <xs:complexType>
          <xs:sequence>
            <xs:element minOccurs="0" maxOccurs="1" name="subject"
             nillable="true" type="xs:string" />
            <xs:element minOccurs="1" maxOccurs="1" name="message"
             type="xs:string" />
            <xs:element minOccurs="1" maxOccurs="1"
             name="recipient" type="xs:string" />
            <xs:element minOccurs="0" maxOccurs="1" name="sender"
             nillable="true" type="xs:string" />
          </xs:sequence>
        </xs:complexType>
      </xs:element>
      <xs:element name="sendEmailResponse">
        <xs:complexType>
          <xs:sequence>
            <xs:element minOccurs="1" maxOccurs="1" name="return"
             type="xs:boolean" />
          </xs:sequence>
        </xs:complexType>
      </xs:element>
    </xs:schema>
  </wsdl:types>

  <wsdl:message name="sendEmailRequest">
    <wsdl:part name="parameters" element="ns:sendEmail" />
  </wsdl:message>
  <wsdl:message name="sendEmailResponse">
    <wsdl:part name="parameters" element="ns:sendEmailResponse" />
  </wsdl:message>

  <wsdl:portType name="eMailWSPortType">
    <wsdl:operation name="sendEmail">
      <wsdl:input message="ns:sendEmailRequest"
       wsaw:Action="urn:sendEmail" />
      <wsdl:output message="ns:sendEmailResponse"
       wsaw:Action="urn:sendEmailResponse" />
    </wsdl:operation>
  </wsdl:portType>

  <!-- Concrete Bindings -->
  <wsdl:binding name="eMailWSSoap12Binding"
   type="ns:eMailWSPortType">
    <soap:binding
     transport="http://schemas.xmlsoap.org/soap/http"
     style="document" />
    <wsdl:operation name="sendEmail">
      <soap:operation soapAction="urn:sendEmail"
       style="document" />
      <wsdl:input>
        <soap:body use="literal" />
      </wsdl:input>
      <wsdl:output>
        <soap:body use="literal" />
      </wsdl:output>
    </wsdl:operation>
  </wsdl:binding>

  <!-- TODO: Adapt the following for deployment -->
  <wsdl:service name="eMailWS">
    <wsdl:port name="eMailWSHttpSoap12Endpoint"
     binding="ns:eMailWSSoap12Binding">
      <soap:address
       location="http://localhost/eMailWS.Soap12Endpoint/" />
    </wsdl:port>
  </wsdl:service>
</wsdl:definitions>
    \end{verbatim}

  \label{soap}
  \subsection{SOAP}
  \nocite{wk_soap}
    %TODO: short intro, more detailed description

  \titlerunning{Web Services - Java und Web Services}
  \label{wsj}
  \section{Java und Web Services}

  \label{jsr224}
  \subsection{JAX-WS 2.0 (JSR-224)}
    Bei JAX-WS\cite{jsr_224} (Java API for XML-Based Web Services) handelt es sich um den Nachfolger von JAX-RPC (JSR-101). JAX-RPC war für RPC-orientierte Web Services konzipiert.

    JAX-WS definiert unter anderem Standard WSDL 1.1 $\Leftrightarrow$ Java Mappings, Standard SOAP- und HTTP-Bindings, ein Standard Handler-Framework und die Client-, Server und Core-APIs für JAX-WS-konforme Web Service-Im\-ple\-men\-tier\-ung\-en.

  \label{jsr181}
  \subsection{WS-Metadata 2.0 (JSR-181)}
  \nocite{jsr_181}

    Es folgt eine Liste der wichtigsten Annotationen mit ihren wichtigsten Parametern und deren Standard-Werten. Die Beschreibungen sind möglichst kurz gehalten. Genauere Details sind in der JSR-181-Spezifikation\cite{jsr_181} zu finden.

    %TODO: insert rest
    %FIXME: make look good
    \texttt{@WebService}:\\
    \begin{tabular}{|l|p{175pt}|p{87pt}|}
    \hline
    \textbf{Parameter} & \textbf{Zweck} & \textbf{Standard} \\
    \hline
    name              & × & Name der Klasse/\linebreak[0]des Interface \\
    \hline
    serviceName       & × & Name der Klasse + \germanquote{Ser\-vice} \\
    \hline
    portName          & × & \texttt{@WebService.name} + \germanquote{Port} \\
    \hline
    targetNamespace   & × & Im\-ple\-men\-tier\-ungs\-ab\-häng\-ig, siehe JAX-WS\cite{jsr_224}, Abschnitt 3.2 \\
    \hline
    wsdlLocation      & × & nichts \\
    \hline
    endpointInterface & × & nichts\newline \newline Wird bei Bedarf im\-ple\-men\-tier\-ungs\-ab\-häng\-ig generiert, wenn es die Zielumgebung erfordert \\
    \hline
    \end{tabular}



  \label{jsr222}
  \subsection{JAXB 2.0 (JSR-222)}
  JAXB\cite{jsr_222} definiert Methoden zum Binding von Java-Klassen auf generierte XML-Schemas.

  \label{summary}
  \titlerunning{Zusammenfassung}
  \section{Zusammenfassung}
  In \decidr{} sollten wir folgende Spezifikationen in den angegebenen Versionen verwenden:
  \begin{itemize}
    \item JAX-WS 2.0
    \item JAXB 2.0
    \item SOAP 1.2
    \item WS-Metadata 2.0
    \item WSDL 1.1
  \end{itemize}


  \newpage
  \titlerunning{Zusammenfassung}
  \addcontentsline{toc}{section}{Literatur}
  \bibliography{literatur}
  \bibliographystyle{splncs}


\end{document}
