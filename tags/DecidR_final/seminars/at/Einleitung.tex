\section{Einleitung}

Heutzutage sind Softwareprojekte im gr��eren Umfang keine Seltenheit mehr. Dabei ist es wichtig einen �berblick �ber die Dateien und ihre �nderungen zu haben. Das �bernimmt die Versionskontrolle Subversion. Mit ihrer Hilfe ist die M�glichkeit geboten stets mit der aktuellsten Version einer Datei zu arbeiten. Durch Subversion k�nnen mehrere Projektlinien des gleichen Hauptprojektes gleichzeitig verfolgt werden. Dabei entstehen keine Komplikationen innerhalb des Projektes. Im Fehlerfall k�nnen vorgenommene �nderungen zur�ckgenommen werden, indem man zu einer vorherigen Version zur�ckkehrt.

Durch das Arbeiten in gro�en Teams ist nicht immer eine gemeinsame Projektstruktur gegeben, da unterschiedliche Entwickler unterschiedliche Strukturen erstellen. Dadurch kann es zu nicht einheitlichen Ordnerstrukturen kommen, was zu unterschiedlichen Buidlfiles f�hrt. Mit Maven wird ein einheitliches Build-Tool bereitgestellt, welches den Entwicklern die M�glichkeit bietet, auf einem gemeinsamen Standard die Implementierung abzuarbeiten. 

Diese beiden Themen werden in Bezug auf das Studienprojekt 2008/2009 \glqq DecidR\grqq{} vorgestellt und sollen zudem den \glqq DecidR-Teammitgliedern\grqq{} vermittelt werden, damit diese ein Verst�ndnis f�r die beiden Werkzeuge Subversion und Maven bekommen. Diese Werkzeuge sollten nach M�glichkeit im Projekt angewendet werden. In Abschnitt 2 wird das Werkzeug \glqq Subversion\grqq{} vorgestellt. Daran schlie�t sich Abschnitt 3 an, in dem Maven beschrieben wird. Den Abschluss der Arbeit bildet die Zusammenfassung in Abschnitt 4.