\subsection{Die Phasen Mavens}

\begin{center}
% use packages: array
\begin{description}
  \item[Kompilierungsphase] In dieser Phase wird das Projekt kompiliert. Beim ersten Kompilieren, werden alle n�tigen Plugins heruntergeladen. Danach greift Maven auf das lokale System zu, falls es Plugins ben�tigt und sucht dort nach ihnen. Maven kompiliert die Klassen, die im Standard-Verzeichnis liegen. Die Klassen werden dort beim Erstellen des Projektes hineingelegt. 
 \item[Testphase] In dieser Phase werden die generierten Testklassen kompiliert und ausgef�hrt. Dabei werden zus�tzliche Plugins f�r das Kompilieren der Testklassen heruntergeladen. Bevor die Testklassen ausgef�hrt werden, wird der Maincode kompiliert.
 \item[Paketierungsphase] In dieser Phase wird aus dem Quellcode ein Paket erstellt. In der \texttt{pom.xml} kann man angeben, was f�r eine Art von Paket (jar, exe etc.) in dieser Phase erstellt werden soll. Die Datei wird in einem standardisiertem Verzeichnis abgelegt. 
 \item[Integrations-Test] In dieser Phase wird durch ein Plugin das fertige Paket in eine Umgebung integriert und getestet. 
 \item[Verifizierungsphase] In dieser Phase wird das Paket �berpr�ft, ob es bestimmte Dateien/Ordner enth�lt und pr�ft den Inhalt auf ihre Struktur. 
 \item[Installationsphase] In dieser Phase wird das Paket in das lokale Maven-Repository installiert. 
 \item[Deployment] Ein Plugin f�r Maven sorgt daf�r, dass das erstellte Paket auf einem entfernten Repository gelegt wird, sodass andere Entwickler das Paket nutzen k�nnen. Alle n�tigen Informationen werden aus der \texttt{pom.xml} entnommen, sodass ein problemloses Deployment stattfinden kann. 
\end{description}
\end{center}
