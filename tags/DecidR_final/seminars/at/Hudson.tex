\subsection{Hudson continuous integration}

Hudson \marginpar{Definition} \cite{hudson} ist ein erweiterbares, webbasiertes Tool zur kontinuierlichen Integration in agilen Softwareprojekten. Kontinuierliche Integration bedeutet eine fortlaufende, permanente Integration der den Prozess des vollst�ndigen Neubildens und Testens einer Anwendung beschreibt. Dabei spielt das agile Softwareprojekt \cite{agil} eine bedeutende Rolle. Denn durch die Agilit�t der Software wird die permanente Integration m�glich. Das Extreme Programming \cite{agil} ist ein Beispiel f�r solch eine Agilit�t. Sobald �nderungen in der Anwendung vorgenommen werden, wird die komplette Anwendung neu gebaut und automatisch getestet. Falls dieser Test erfolgreich ist, wird die Anwendung in die n�chste Stufe gereicht. Falls der Test fehl schl�gt, findet ein Rollback statt und die Entwickler werden aufgefordert die Anwendung zu verbessern. Und genau das �bernimmt Hudson, in automatisierter Form. 

Hudson \marginpar{Funktionalit�ten} wurde in erster Linie von Kohsuke Kawaguchi, Mitarbeiter von Sun Microsystems, entwickelt. Bis auf die Icons steht das komplette Programm unter der MIT-Lizenz \cite{hudson}. Hudson ist in Java geschrieben und l�uft auf einem beliebigen Servlet-Container. Es werden s�mtliche g�ngigen Build-Tools unterst�zt, wie Apache-Ant oder Apache-Maven. Au�erdem kann Hudson den Quellcode automatisch aus einem SVN- oder CVS-Repository auschecken.
Durch Plugins ist Hudson erweiterbar, diese bieten zum Beispiel eine Anbindung an \glqq  GoogleCode \grqq{}. Desweiteren bietet Hudson die M�glichkeit, �nderungen per Email zu verschicken und so die betreffenden Entwickler zu informieren. 

Dabei \marginpar{Vorteile} steht die Einfachheit an erster Stelle. Gerade in Bezug auf Maven bietet Hudson eine sehr einfache und praktikable M�glichkeit. Es ist m�glich ein Maven Projekt zu erstellen, sodass Hudson es �berwacht und �nderungen zur sofortigen Neubildung und Testen der Anwendung f�hrt. Durch die kontinuierliche Integration, die Hudson bietet, wird die Programmierung vorteilhafter. Dabei werden
\begin{itemize}
 \item Fehler laufend entdeckt und behoben und nicht erst vor den Meilenstein
 \item fr�he Warnungen bei nicht passenden Bestandteilen ausgegeben
 \item Fehler bei Unit-Tests sofort entdeckt
 \item lauff�hige Versionen f�r Test-, Demo- und Vertriebszwecke angeboten
 \item die Entwickler gezwungen verwantwortungsvoller mit ihren T�tigkeiten umzugehen, um Fehler zu vermeiden
\end{itemize}
